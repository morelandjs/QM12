\documentclass[a4paper,10pt]{elsarticle}
\usepackage[utf8x]{inputenc}

%opening
\title{Imprinting Quantum Fluctuations on  \\ Hydrodynamic Initial Conditions}
\author{J. S. Moreland, Z. Qiu, U. Heinz}

\begin{document}



\begin{abstract}
  Starting from the two-point covariance function derived in [1], we have developed a toy model to investigate the effect of gluonic 
fluctuations on the transverse energy profile predicted by Color-Glass Condensate initial conditions. We find that the effect of these fluctuations
on the eccentricity harmonics $\epsilon_{n}$ varies strongly with the correlation length and consequently the value of the fixed saturation momentum $Q_{s}$ 
used in the M\"uller-Sch\"afer calculation. Using approximate values for the minimum and maximum saturation momenta probed at RHIC in central Au-Au collisions, 
we estimate the increase in the eccentricity coefficients resulting from gluonic fluctuations is of order $10-20\%$ at RHIC in central collisions.

\end{abstract}

\maketitle

\section{Introduction}
Hybrid models, which couple viscous hydrodynamics to a Boltzmann cascade, have been highly successfully in describing collective flow properties of 
the quark-gluon plasma (QGP) and subsequent hadron resonance gas produced in relativistic heavy-ion collisions. The success of these simulations
has generated accute interest in the hydrodynamic transport properties of the produced medium, specifically the QGP shear viscosity to entropy ratio $\eta/s$. 
Phenomenological extractions of $\eta/s$ typically exploit the one-to-one mapping $\epsilon_{n}\leftrightarrow v_{n}$ between the initial state eccentricity harmonics $\epsilon_{n}$
characterized by,
\begin{equation}
 \epsilon_{n} e^{i n \Phi_n}= - \frac{\int r ~\!dr ~\!d\phi ~\!r^2 ~\!e^{i n \phi}\rho(r,\phi)}{   \int r ~\!dr ~\!d\phi ~\!e^{i n \phi} ~\! \rho(r,\phi)}
\end{equation}
and the final state azimuthal flow harmonics $v_{n}$ characterized by,
\begin{equation}
 v_{n} e^{i n \Psi_n}=\frac{\int p_T ~\!dp_T ~\!d\phi_p ~\!e^{i n \phi_p}\frac{dN_{ch}}{d\eta p_T dp_T d\phi_p}}{\int p_T ~\!dp_T ~\!d\phi_p \frac{dN_{ch}}{d\eta p_T dp_T d\phi_p}}
\end{equation}
Uncertainties in the models used to compute the initial state eccentricity coefficients $\epsilon_n$ are currently the largest source of error in phenomenological extractions of $\eta/s$, and 
stricter constraints on $\eta/s$ require more more realistic descriptions of the initial state geometry.

\section{From nucleonic to sub-nucleonic fluctuations}

The importance of event-by-event fluctuations in the distribution of participant nucleons was first pointed out by Miller and Snellings.These fluctuations correctly explain
the existence of odd flow harmonics $v_{2n+1}$ and the non-vanishing anisotropic flow detected in central collisions $v_n | _{b=0} >0$. Numerous Monte Carlo implementations 
of these event-by-event fluctuations in both Glauber and Color-Glass Condensate frameworks. These fluctuations



\begin{center}
 \includegraphics[scale=0.35]{../../../Research/QM2012Talk/ppFluct/ppFluct.pdf}
 % ppFluct.pdf: 792x612 pixel, 72dpi, 27.94x21.59 cm, bb=0 0 792 612
\end{center}

\section{Generating a toy model for transverse gluonic field fluctuations using the M\"uller-Sch\"afer Covariance}



\section{Results and Conclusions}

\end{document}
